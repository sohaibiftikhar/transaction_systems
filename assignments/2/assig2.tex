\documentclass[
%journal=ancac3, % for ACS Nano
%journal=acbcct, % for ACS Chem. Biol.
journal=jacsat, % for undefined journal
manuscript=article]{achemso}

\usepackage[version=3]{mhchem} % Formula subscripts using \ce{}
\usepackage{mathtools}

\newcommand*{\mycommand}[1]{\texttt{\emph{#1}}}

\author{Sohaib Iftikhar}
\email{sohaib.iftikhar@tum.de}
\title[\texttt{achemso} demonstration]
{Transaction Systems Assignment 2}

\begin{document}
\begin{center}
Matrkl No: 03681005
\end{center}

\section{Exercise 1}
Given the schedule and that transactions 3 and 4 copiers. 
\[s = r_{1}(x) r_{3}(x) w_{3}(y) w_{2}(x) c_{3} r_{4}(y) w_{4}(x) c_{2} r_{5}(x) c_{4} w_{5}(z) w_{1}(z) c_{1} c_{5}\]
Calculate the Herbert semantics.
\subsection{Solution}
% \tag*{Pythagoras}
\[H_{s}(r_{1}(x)) = f_{0x}()\]
\[H_{s}(r_{3}(x)) = f_{0x}()\]
\[H_{s}(w_{3}(y)) = f_{0y}() \tag*{Copier}\]
\[H_{s}(w_{2}(x)) = f_{2x}() \tag*{No Input}\]
\[H_{s}(r_{4}(y)) = H_{s}(w_{3}(y)) = f_{0y}()\]
\[H_{s}(w_{4}(x)) = f_{2x}() \tag*{Copier}\]
\[H_{s}(r_{5}(x)) = H_{s}(w_{4}(x) = f_{2x}()\]
\[H_{s}(w_{5}(z)) = f_{5z}(H_{s}(r_{5}(x))) = f_{5z}(f_{2x}())\]
\[H_{s}(w_{1}(z)) = f_{1z}(H_{s}(r_{1}(x))) = f_{1z}(f_{0x}())\]
\\
Using these we can calculate the semantics for the schedule s as:
\[H_{s}\{s\}(x) = H_{s}(w_{4}(x)) = f_{2x}()\]
\[H_{s}\{s\}(y) = H_{s}(w_{3}(y)) = f_{0y}()\]
\[H_{s}\{s\}(z) = H_{s}(w_{1}(z)) = f_{1z}(f_{0x}())\]

\section{Exercise 2}
Given the schedule
\[s = r_{1}(x) r_{3}(x) w_{3}(y) w_{2}(x) r_{4}(y) c_{2} w_{4}(x) c_{4} r_{5}(x) c_{3} w_{5}(z) c_{5} w_{1}(z) c_{1}\]
Does it belong to FSR? CSR?
\subsection{Solution}
The schedule is the same as the previous exercise except there are not copiers. We first calculate the Herbert semantics for this schedule.
\[H_{s}(r_{1}(x)) = f_{0x}()\]
\[H_{s}(r_{3}(x)) = f_{0x}()\]
\[H_{s}(w_{3}(y)) = f_{3y}(f_{0x}())\]
\[H_{s}(w_{2}(x)) = f_{2x}()\]
\[H_{s}(r_{4}(y)) = H_{s}(w_{3}(y)) = f_{3y}(f_{0x}())\]
\[H_{s}(w_{4}(x)) = f_{4x}(H_s(r_{4}(y)) = f_{4x}(f_{3y}(f_{0x}()))\]
\[H_{s}(r_{5}(x)) = H_{s}(w_{4}(x) = f_{4x}(f_{3y}(f_{0x}()))\]
\[H_{s}(w_{5}(z)) = f_{5z}(H_{s}(r_{5}(x))) = f_{5z}(f_{4x}(f_{3y}(f_{0x}())))\]
\[H_{s}(w_{1}(z)) = f_{1z}(H_{s}(r_{1}(x))) = f_{1z}(f_{0x}())\]
\\
Using these we can calculate the semantics for the schedule s as:
\[H_{s}\{s\}(x) = H_{s}(w_{4}(x)) = f_{4x}(f_{3y}(f_{0x}()))\]
\[H_{s}\{s\}(y) = H_{s}(w_{3}(y)) = f_{3y}(f_{0x}())\]
\[H_{s}\{s\}(z) = H_{s}(w_{1}(z)) = f_{1z}(f_{0x}())\]
\\
Looking at this we can say $T_2$ and $T_5$ are not needed for the final states. As a result they must be placed in such a way that they do not alter the result. Since $T_2$ writes $x$ but $T_3$ and $T_1$ read the original $x$ for their output $T_1$ and $T_3$ must occur before $T_2$. Now $T_4$ depends on $T_3$ for input for variable $y$ which means $T_3$ must occur before $T_4$. Finally $T_5$ must finish before $T_1$ as $T_1$ must override the value of $z$ written by $T_5$. We can conclude that the following sequence for the transaction 
$T_{5}T_{1}T_{3}T_{2}T_{4}$ maybe produce the same Herbert semantics. We can verify this now with the new schedule.
\[ s' = r_{5}(x) w_{5}(z) c_{5} r_{1}(x) w_{1}(z) c_{1} r_{3}(x) w_{3}(y) c_{3} w_{2}(x) c_{2} r_{4}(y) w_{4}(x) \]
We calculate the Herbert semantics for this new schedule:
\[
H_{s}\{s\}(x) = H_{s}(w_{4}(x)) = f_{4x}(H_{s}(r_{4}(y))) = f_{4x}(H_s(w_3{y})) = f_{4x}(f_{3y}(H_s(r_{3}(x)))) = f_{4x}(f_{3y}(f_{0x}()))
\]
\[ H_{s}\{s\}(y) = H_{s}(w_{3}(y)) = f_{3y}(H_s(r_{3}(x))) = f_{3y}(f_{0x}()) \]
\[ H_{s}\{s\}(z) = H_{s}(w_{1}(z)) = f_{1z}(H_{s}(r_{1}(x))) = f_{1z}(f_{0x}()) \]

Now by the definition of FSR since the Herbert semantics of the two sequences are equivalent the schedule is FSR.
\end{document}
